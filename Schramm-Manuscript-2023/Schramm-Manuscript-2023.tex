%  LaTeX support: latex@mdpi.com
%  For support, please attach all files needed for compiling as well as the log file, and specify your operating system, LaTeX version, and LaTeX editor.

%=================================================================
% pandoc conditionals added to preserve backwards compatibility with previous versions of rticles
\documentclass[water,article,submit,oneauthor]{Definitions/mdpi}

% If you would like to post an early version of this manuscript as a preprint, you may use preprint as the journal and change 'submit' to 'accept'. The document class line would be, e.g., \documentclass[preprints,article,accept,moreauthors,pdftex]{mdpi}. This is especially recommended for submission to arXiv, where line numbers should be removed before posting. For preprints.org, the editorial staff will make this change immediately prior to posting.

%% Some pieces required from the pandoc template
\setlist[itemize]{leftmargin=*,labelsep=5.8mm}
\setlist[enumerate]{leftmargin=*,labelsep=4.9mm}


%--------------------
% Class Options:
%--------------------
%----------
% journal
%----------
% Choose between the following MDPI journals:
% acoustics, actuators, addictions, admsci, adolescents, aerobiology, aerospace, agriculture, agriengineering, agrochemicals, agronomy, ai, air, algorithms, allergies, alloys, analytica, analytics, anatomia, animals, antibiotics, antibodies, antioxidants, applbiosci, appliedchem, appliedmath, applmech, applmicrobiol, applnano, applsci, aquacj, architecture, arm, arthropoda, arts, asc, asi, astronomy, atmosphere, atoms, audiolres, automation, axioms, bacteria, batteries, bdcc, behavsci, beverages, biochem, bioengineering, biologics, biology, biomass, biomechanics, biomed, biomedicines, biomedinformatics, biomimetics, biomolecules, biophysica, biosensors, biotech, birds, bloods, blsf, brainsci, breath, buildings, businesses, cancers, carbon, cardiogenetics, catalysts, cells, ceramics, challenges, chemengineering, chemistry, chemosensors, chemproc, children, chips, cimb, civileng, cleantechnol, climate, clinpract, clockssleep, cmd, coasts, coatings, colloids, colorants, commodities, compounds, computation, computers, condensedmatter, conservation, constrmater, cosmetics, covid, crops, cryptography, crystals, csmf, ctn, curroncol, cyber, dairy, data, ddc, dentistry, dermato, dermatopathology, designs, devices, diabetology, diagnostics, dietetics, digital, disabilities, diseases, diversity, dna, drones, dynamics, earth, ebj, ecologies, econometrics, economies, education, ejihpe, electricity, electrochem, electronicmat, electronics, encyclopedia, endocrines, energies, eng, engproc, entomology, entropy, environments, environsciproc, epidemiologia, epigenomes, est, fermentation, fibers, fintech, fire, fishes, fluids, foods, forecasting, forensicsci, forests, foundations, fractalfract, fuels, future, futureinternet, futurepharmacol, futurephys, futuretransp, galaxies, games, gases, gastroent, gastrointestdisord, gels, genealogy, genes, geographies, geohazards, geomatics, geosciences, geotechnics, geriatrics, grasses, gucdd, hazardousmatters, healthcare, hearts, hemato, hematolrep, heritage, higheredu, highthroughput, histories, horticulturae, hospitals, humanities, humans, hydrobiology, hydrogen, hydrology, hygiene, idr, ijerph, ijfs, ijgi, ijms, ijns, ijpb, ijtm, ijtpp, ime, immuno, informatics, information, infrastructures, inorganics, insects, instruments, inventions, iot, j, jal, jcdd, jcm, jcp, jcs, jcto, jdb, jeta, jfb, jfmk, jimaging, jintelligence, jlpea, jmmp, jmp, jmse, jne, jnt, jof, joitmc, jor, journalmedia, jox, jpm, jrfm, jsan, jtaer, jvd, jzbg, kidneydial, kinasesphosphatases, knowledge, land, languages, laws, life, liquids, literature, livers, logics, logistics, lubricants, lymphatics, machines, macromol, magnetism, magnetochemistry, make, marinedrugs, materials, materproc, mathematics, mca, measurements, medicina, medicines, medsci, membranes, merits, metabolites, metals, meteorology, methane, metrology, micro, microarrays, microbiolres, micromachines, microorganisms, microplastics, minerals, mining, modelling, molbank, molecules, mps, msf, mti, muscles, nanoenergyadv, nanomanufacturing,\gdef\@continuouspages{yes}} nanomaterials, ncrna, ndt, network, neuroglia, neurolint, neurosci, nitrogen, notspecified, %%nri, nursrep, nutraceuticals, nutrients, obesities, oceans, ohbm, onco, %oncopathology, optics, oral, organics, organoids, osteology, oxygen, parasites, parasitologia, particles, pathogens, pathophysiology, pediatrrep, pharmaceuticals, pharmaceutics, pharmacoepidemiology,\gdef\@ISSN{2813-0618}\gdef\@continuous pharmacy, philosophies, photochem, photonics, phycology, physchem, physics, physiologia, plants, plasma, platforms, pollutants, polymers, polysaccharides, poultry, powders, preprints, proceedings, processes, prosthesis, proteomes, psf, psych, psychiatryint, psychoactives, publications, quantumrep, quaternary, qubs, radiation, reactions, receptors, recycling, regeneration, religions, remotesensing, reports, reprodmed, resources, rheumato, risks, robotics, ruminants, safety, sci, scipharm, sclerosis, seeds, sensors, separations, sexes, signals, sinusitis, skins, smartcities, sna, societies, socsci, software, soilsystems, solar, solids, spectroscj, sports, standards, stats, std, stresses, surfaces, surgeries, suschem, sustainability, symmetry, synbio, systems, targets, taxonomy, technologies, telecom, test, textiles, thalassrep, thermo, tomography, tourismhosp, toxics, toxins, transplantology, transportation, traumacare, traumas, tropicalmed, universe, urbansci, uro, vaccines, vehicles, venereology, vetsci, vibration, virtualworlds, viruses, vision, waste, water, wem, wevj, wind, women, world, youth, zoonoticdis
% For posting an early version of this manuscript as a preprint, you may use "preprints" as the journal. Changing "submit" to "accept" before posting will remove line numbers.

%---------
% article
%---------
% The default type of manuscript is "article", but can be replaced by:
% abstract, addendum, article, book, bookreview, briefreport, casereport, comment, commentary, communication, conferenceproceedings, correction, conferencereport, entry, expressionofconcern, extendedabstract, datadescriptor, editorial, essay, erratum, hypothesis, interestingimage, obituary, opinion, projectreport, reply, retraction, review, perspective, protocol, shortnote, studyprotocol, systematicreview, supfile, technicalnote, viewpoint, guidelines, registeredreport, tutorial
% supfile = supplementary materials

%----------
% submit
%----------
% The class option "submit" will be changed to "accept" by the Editorial Office when the paper is accepted. This will only make changes to the frontpage (e.g., the logo of the journal will get visible), the headings, and the copyright information. Also, line numbering will be removed. Journal info and pagination for accepted papers will also be assigned by the Editorial Office.

%------------------
% moreauthors
%------------------
% If there is only one author the class option oneauthor should be used. Otherwise use the class option moreauthors.

%---------
% pdftex
%---------
% The option pdftex is for use with pdfLaTeX. Remove "pdftex" for (1) compiling with LaTeX & dvi2pdf (if eps figures are used) or for (2) compiling with XeLaTeX.

%=================================================================
% MDPI internal commands - do not modify
\firstpage{1}
\makeatletter
\setcounter{page}{\@firstpage}
\makeatother
\pubvolume{1}
\issuenum{1}
\articlenumber{0}
\pubyear{2023}
\copyrightyear{2023}
%\externaleditor{Academic Editor: Firstname Lastname}
\datereceived{ }
\daterevised{ } % Comment out if no revised date
\dateaccepted{ }
\datepublished{ }
%\datecorrected{} % For corrected papers: "Corrected: XXX" date in the original paper.
%\dateretracted{} % For corrected papers: "Retracted: XXX" date in the original paper.
\hreflink{https://doi.org/} % If needed use \linebreak
%\doinum{}
%\pdfoutput=1 % Uncommented for upload to arXiv.org

%=================================================================
% Add packages and commands here. The following packages are loaded in our class file: fontenc, inputenc, calc, indentfirst, fancyhdr, graphicx, epstopdf, lastpage, ifthen, float, amsmath, amssymb, lineno, setspace, enumitem, mathpazo, booktabs, titlesec, etoolbox, tabto, xcolor, colortbl, soul, multirow, microtype, tikz, totcount, changepage, attrib, upgreek, array, tabularx, pbox, ragged2e, tocloft, marginnote, marginfix, enotez, amsthm, natbib, hyperref, cleveref, scrextend, url, geometry, newfloat, caption, draftwatermark, seqsplit
% cleveref: load \crefname definitions after \begin{document}

%=================================================================
% Please use the following mathematics environments: Theorem, Lemma, Corollary, Proposition, Characterization, Property, Problem, Example, ExamplesandDefinitions, Hypothesis, Remark, Definition, Notation, Assumption
%% For proofs, please use the proof environment (the amsthm package is loaded by the MDPI class).

%=================================================================
% Full title of the paper (Capitalized)
\Title{Assessing linkages between watershed nutrient loading and estuary
water quality in Lavaca Bay, Texas}

% MDPI internal command: Title for citation in the left column
\TitleCitation{Assessing linkages between watershed nutrient loading and
estuary water quality in Lavaca Bay, Texas}

% Author Orchid ID: enter ID or remove command
%\newcommand{\orcidauthorA}{0000-0000-0000-000X} % Add \orcidA{} behind the author's name
%\newcommand{\orcidauthorB}{0000-0000-0000-000X} % Add \orcidB{} behind the author's name


% Authors, for the paper (add full first names)
\Author{Michael
Schramm$^{1}$\href{https://orcid.org/0000-0003-1876-6592}
{\orcidicon}}


%\longauthorlist{yes}


% MDPI internal command: Authors, for metadata in PDF
\AuthorNames{Michael Schramm}

% MDPI internal command: Authors, for citation in the left column
%\AuthorCitation{Lastname, F.; Lastname, F.; Lastname, F.}
% If this is a Chicago style journal: Lastname, Firstname, Firstname Lastname, and Firstname Lastname.
\AuthorCitation{Schramm, M.}

% Affiliations / Addresses (Add [1] after \address if there is only one affiliation.)
\address{%
$^{1}$ \quad Texas A\&M AgriLife Research - Texas Water Resources
Institute 1001 Holleman Dr.~E. College Station, TX
77840-2118; \href{mailto:michael.schramm@ag.tamu.edu}{\nolinkurl{michael.schramm@ag.tamu.edu}}\\
}

% Contact information of the corresponding author
\corres{Correspondence: \href{mailto:michael.schramm@ag.tamu.edu}{\nolinkurl{michael.schramm@ag.tamu.edu}}}

% Current address and/or shared authorship








% The commands \thirdnote{} till \eighthnote{} are available for further notes

% Simple summary

%\conference{} % An extended version of a conference paper

% Abstract (Do not insert blank lines, i.e. \\)
\abstract{Lavaca Bay is a small secondary embayment on the Texas coast
that is displaying early signals of water quality degradation. This
study uses Generalized Additive Models to assesses both watershed
nutrient loading and responses in estuary water quality to nutrient
loading and streamflow. While delivered nutrient loads varied greatly,
flow-normalized nutrient loads indicated limited evidence of changes in
nutrient runoff in the watershed since 2005. As a consequence, there
wasn't strong evidence for an effect of nutrient loading (controlled for
flow) on dissolved oxygen or chlorophyll-\emph{a} concentrations. There
was evidence total phosphorus (TP) loading had a strong positive effect
on TP concentration in the Bay, particularly at sites closest to the
river mouth. Seasonally-adjusted streamflow had strong effects on
nutrient concentration and chlorophyll-\emph{a} concentrations at most
sites. Although temporal trends did not identify significant estuary
water quality degredation in dissolved oxygen or chlorophyll-\emph{a}
concentrations, site specific trends in both organic and inorganic
nitrogen are concerning for the long-term health of the estuary. This
study provides a baseline assessment of the linkages between watershed
nutrient loading and water quality responses in Lavaca Bay. Some of the
data gaps highlighted in the study are being addressed in ongoing
projects to provide improved information for resource managers.}


% Keywords
\keyword{estuary; nutrient loading; water quality; Texas}

% The fields PACS, MSC, and JEL may be left empty or commented out if not applicable
%\PACS{J0101}
%\MSC{}
%\JEL{}

%%%%%%%%%%%%%%%%%%%%%%%%%%%%%%%%%%%%%%%%%%
% Only for the journal Diversity
%\LSID{\url{http://}}

%%%%%%%%%%%%%%%%%%%%%%%%%%%%%%%%%%%%%%%%%%
% Only for the journal Applied Sciences
%\featuredapplication{Authors are encouraged to provide a concise description of the specific application or a potential application of the work. This section is not mandatory.}
%%%%%%%%%%%%%%%%%%%%%%%%%%%%%%%%%%%%%%%%%%

%%%%%%%%%%%%%%%%%%%%%%%%%%%%%%%%%%%%%%%%%%
% Only for the journal Data
%\dataset{DOI number or link to the deposited data set if the data set is published separately. If the data set shall be published as a supplement to this paper, this field will be filled by the journal editors. In this case, please submit the data set as a supplement.}
%\datasetlicense{License under which the data set is made available (CC0, CC-BY, CC-BY-SA, CC-BY-NC, etc.)}

%%%%%%%%%%%%%%%%%%%%%%%%%%%%%%%%%%%%%%%%%%
% Only for the journal Toxins
%\keycontribution{The breakthroughs or highlights of the manuscript. Authors can write one or two sentences to describe the most important part of the paper.}

%%%%%%%%%%%%%%%%%%%%%%%%%%%%%%%%%%%%%%%%%%
% Only for the journal Encyclopedia
%\encyclopediadef{For entry manuscripts only: please provide a brief overview of the entry title instead of an abstract.}

%%%%%%%%%%%%%%%%%%%%%%%%%%%%%%%%%%%%%%%%%%
% Only for the journal Advances in Respiratory Medicine
%\addhighlights{yes}
%\renewcommand{\addhighlights}{%

%\noindent This is an obligatory section in “Advances in Respiratory Medicine”, whose goal is to increase the discoverability and readability of the article via search engines and other scholars. Highlights should not be a copy of the abstract, but a simple text allowing the reader to quickly and simplified find out what the article is about and what can be cited from it. Each of these parts should be devoted up to 2~bullet points.\vspace{3pt}\\
%\textbf{What are the main findings?}
% \begin{itemize}[labelsep=2.5mm,topsep=-3pt]
% \item First bullet.
% \item Second bullet.
% \end{itemize}\vspace{3pt}
%\textbf{What is the implication of the main finding?}
% \begin{itemize}[labelsep=2.5mm,topsep=-3pt]
% \item First bullet.
% \item Second bullet.
% \end{itemize}
%}

%%%%%%%%%%%%%%%%%%%%%%%%%%%%%%%%%%%%%%%%%%


% tightlist command for lists without linebreak
\providecommand{\tightlist}{%
  \setlength{\itemsep}{0pt}\setlength{\parskip}{0pt}}



\usepackage{booktabs}
\usepackage{longtable}
\usepackage{array}
\usepackage{multirow}
\usepackage{wrapfig}
\usepackage{float}
\usepackage{colortbl}
\usepackage{pdflscape}
\usepackage{tabu}
\usepackage{threeparttable}
\usepackage{threeparttablex}
\usepackage[normalem]{ulem}
\usepackage{makecell}
\usepackage{xcolor}
\usepackage{siunitx}

  \newcolumntype{d}{S[
    input-open-uncertainty=,
    input-close-uncertainty=,
    parse-numbers = false,
    table-align-text-pre=false,
    table-align-text-post=false
  ]}
  

\begin{document}



%%%%%%%%%%%%%%%%%%%%%%%%%%%%%%%%%%%%%%%%%%

\hypertarget{introduction}{%
\section{Introduction}\label{introduction}}

Like many coastal areas globally, coastal watersheds along the Texas
Gulf coast are facing pressures from increasing population, increases in
point source and non-point source pollution and alterations to
freshwater flows that affect water quality in downstream estuaries
\citep{bricker_effects_2008, kennicuttWaterQualityGulf2017, bugica_water_2020}.
Despite these increasing pressures, national scale assessments have
classified coastal estuaries in Texas as moderate or lower for
exhibiting eutrophic conditions \citep{bricker_effects_2008}. However, a
suite of recent studies indicate that estuary water quality dynamics in
both agriculturally dominated and urban watersheds within Texas are in
fact expressing conditions that are increasingly conducive to algal
blooms and eutrophication
\citep{wetzWaterQualityDynamics2016, wetz_exceptionally_2017, bugica_water_2020, chinPhytoplanktonBiomassCommunity2022}.
With identification of localized areas of estuary water quality concern
along the Texas coast \citep{bugica_water_2020}, localized studies are
being prioritized to better inform management actions.

This project provides an assessment of nutrient loading and water
quality responses in Lavaca Bay, Texas. Lavaca Bay is a secondary bay in
the larger Matagorda Bay system located roughly halfway between Houston,
Texas and Corpus Christi, Texas. Lavaca Bay has faced substantial
challenges associated with legacy contamination but general water
quality parameters such as dissolved oxygen (DO), nutrients, and
biological parameters have been well within state water quality
standards. However, long-term declines of benthic fauna abundance,
biomass, and diversity in Lavaca Bay primarily linked to reductions in
freshwater inflows and changes in estuary salinity
\citep{beserespollackLongtermTrendsResponse2011, palmerImpactsDroughtsLow2015, montagnaAssessmentRelationshipFreshwater2020}
are a concern to local stakeholders. \citet{bugica_water_2020}
identified monotonic increases in total phosphorus (TP), orthophosphate,
total Kjeldahl nitrogen (TKN), and chlorophyll-\emph{a} at sites within
Lavaca Bay. Although long-term changes in DO concentrations were not
identified, the trends in nutrient concentrations are concerning due to
the role of nitrogen as a limiting factor for primary production in many
Texas estuaries
\citep{gardnerNitrogenFixationDissimilatory2006, houTransformationFateNitrate2012, doradoUnderstandingInteractionsFreshwater2015, paudelRelationshipSuspendedSolids2019, wetz_exceptionally_2017}
and the ramifications that changes in nitrogen loadings could have for
productivity and eutrophication in Lavaca Bay.

There are ongoing efforts between local, state, and federal agencies to
address water quality impairments in the freshwater portions of the
Lavaca Bay watershed
\citep{jainTechnicalSupportDocument2021, schramm_lavaca_2018, bertholdDirectMailingEducation2021}.
However, at a statewide scale, these approaches have shown limited
success and emphasize a need for improved efforts at assessing and
linking management actions with downstream water quality to better
identify and replicate effective management actions across the state
\citep{schrammTotalMaximumDaily2022}. The identification and
communication of changes and trends in water quality is complicated by
the fact that trends are often non-linear and confounded by
precipitation and runoff that hinder traditional analysis
\citep{wazniakLinkingWaterQuality2007, lloydMethodsDetectingChange2014}.
To provide actionable information for resource managers, water quality
conditions must be evaluated relative to changes in natural
environmental drivers to better understand and manage potential
anthropogenic effects. This study utilizes semiparemetric methods to
develop estimates of delivered and flow-normalized nutrient loads and
assess changes in loads delivered to Lavaca Bay. The study also assesses
the response of water quality parameters in Lavaca Bay over time and in
response to freshwater inflow controlled for seasonality and to
watershed nutrient loads that are controlled for environmentally driven
variation.

\hypertarget{materials-and-methods}{%
\section{Materials and Methods}\label{materials-and-methods}}

\hypertarget{study-area-and-data}{%
\subsection{Study Area and Data}\label{study-area-and-data}}

\begin{figure}
\includegraphics[width=1\linewidth]{Schramm-Manuscript-2023_files/figure-latex/fig1-1.png}
\caption{Map of the Lavaca Bay watershed, location of USGS gages where nutrient loads were calculated, and location of estuary water quality sampling sites.\label{fig:fig1}}
\end{figure}  

Lavaca Bay is 190 km\textsuperscript{2} with the majority of freshwater
inflow provided by the Lavaca and Navidad River systems (Figure
\ref{fig:fig1}). The Garcitas-Arenosa, Placedo Creek, and Cox Bay
watersheds provide additional freshwater inflows. The entire watershed
land area is 8,149 km\textsuperscript{2} and primarily rural. Watershed
land cover and land use is 50\% grazed pasture and rangeland, 20\%
cultivated cropland (primarily rows crops such as corn, cotton, and
sorghum), and 5\% suburban/urban. Pasture and rangeland is concentrated
in the Lavaca River watershed, while cultivated crops are generally
located along the eastern tributaries of the Navidad river. The Lavaca
and Navidad River watersheds are a combined 5,966 km\textsuperscript{2},
or approximately 73\% of the entire Lavaca Bay watershed area. Discharge
from the Navidad River is regulated by Lake Texana which has been in
operation since 1980. Lake Texana provides 0.210 km\textsuperscript{3}
of water storage and discharges into the tidal section of the Navidad
River which ultimately joins the tidal section of the Lavaca River 15 km
upstream of the confluence with the Lavaca Bay.

Daily discharges for the Lavaca River (USGS-08164000, Figure
\ref{fig:fig1}) were obtained from the United States Geologic Survey
(USGS) National Water Information System using the \emph{dataRetrieval}
R package \citep{deciccoDataRetrievalPackagesDiscovering2022}. Gaged
daily discharges from the outlet of Lake Texana on the Navidad River
(USGS-0816425) were provided by the Texas Water Development Board (TWDB)
(April 21, 2022 email from R. Neupane, TWDB).

Water quality sample data for the two freshwater and three estuary
locations were obtained from the Texas Commission on Environmental
Quality (TCEQ) Surface Water Quality Monitoring Information System. Data
submitted through the system are required to be collected under Quality
Assurance Project Plans and lab method procedures outlined by the TCEQ's
procedures manual. The QAPP and procedures manuals ensure the consistent
collection and laboratory methods are applied between samples collected
by different entities and under different projects. All sites had
varying lengths of and availability of data. For freshwater locations,
TP from January 2000 through December 2020 and nitrate-nitrogen
(NO\textsubscript{3}) data from January 2005 through December 2020 were
downloaded (Table \ref{tab:fwsummary}). Less than 5-years of total
nitrogen and TKN concentration data were available at the freshwater
sites and deemed insufficient to develop load estimation models
\citep{horowitzEvaluationSedimentRating2003, snelderEstimationCatchmentNutrient2017}.
The three estuary sites included an upper Lavaca Bay site near the
outlet of the Lavaca River system (TCEQ-13563), a mid-Lavaca Bay site
(TCEQ-13383), and the lower Lavaca Bay site near the mouth of the Bay
(TCEQ-13384). For estuary locations, we obtained data for TP,
Nitrite+Nitrate (NO\emph{\textsubscript{x}}), TKN, chlorophyll-\emph{a},
and DO concentrations from January 2005 through December 2020 (Table
\ref{tab:estuarysummary}).

\begin{table}[H]

\caption{\label{tab:fwsummary}Summary of gauged streamflow and freshwater water quality samples between January 1, 2000 and December 31, 2020.}
\centering
\begin{tabular}[t]{llrrr}
\toprule
Station ID &   & Mean & SD & N\\
\midrule
USGS-08164000 & TP (mg/L) & \num{0.21} & \num{0.09} & 80\\
 & NO\textsubscript{3} (mg/L) & \num{0.18} & \num{0.24} & 74\\
 & Mean Daily Streamflow (cfs) & \num{332.78} & \num{1667.47} & 7671\\
USGS-08164525 & TP (mg/L) & \num{0.20} & \num{0.08} & 81\\
 & NO\textsubscript{3} (mg/L) & \num{0.29} & \num{0.26} & 62\\
 & Mean Daily Streamflow (cfs) & \num{666.14} & \num{2957.79} & 7671\\
\bottomrule
\end{tabular}
\end{table}

\begin{table}[H]

\caption{\label{tab:estuarysummary}Summary of estuary water quality samples collected between January 1, 2005 and December 31, 2020.}
\centering
\begin{tabular}[t]{lllll}
\toprule
Station ID &   & Mean & SD & N\\
\midrule
 & TP (mg/L) & 0.11 & 0.05 & 47\\

 & NO\textsubscript{\emph{x}} (mg/L) & 0.07 & 0.15 & 51\\

 & TKN (mg/L) & 0.94 & 0.49 & 45\\

 & Chlorophyll-\emph{a} ($\mu$g/L) & 9.43 & 5.31 & 47\\

\multirow{-5}{*}{\raggedright\arraybackslash TCEQ-13383} & DO (mg/L) & 7.22 & 1.35 & 55\\
\cmidrule{1-5}
 & TP (mg/L) & 0.08 & 0.03 & 51\\

 & NO\textsubscript{\emph{x}} (mg/L) & 0.06 & 0.08 & 52\\

 & TKN (mg/L) & 0.76 & 0.40 & 48\\

 & Chlorophyll-\emph{a} ($\mu$g/L) & 8.22 & 6.44 & 46\\

\multirow{-5}{*}{\raggedright\arraybackslash TCEQ-13384} & DO (mg/L) & 7.51 & 1.32 & 54\\
\cmidrule{1-5}
 & TP (mg/L) & 0.13 & 0.06 & 50\\

 & NO\textsubscript{\emph{x}} (mg/L) & 0.09 & 0.13 & 53\\

 & TKN (mg/L) & 0.94 & 0.37 & 49\\

 & Chlorophyll-\emph{a} ($\mu$g/L) & 9.67 & 5.33 & 49\\

\multirow{-5}{*}{\raggedright\arraybackslash TCEQ-13563} & DO (mg/L) & 7.91 & 1.34 & 56\\
\bottomrule
\end{tabular}
\end{table}

\hypertarget{estimating-watershed-based-nutrient-loads}{%
\subsection{Estimating Watershed Based Nutrient
Loads}\label{estimating-watershed-based-nutrient-loads}}

Estimates of NO\textsubscript{3} and TP loads at the Lavaca River
(USGS-08164000) and the outlet of Lake Texana on the Navidad River
(USGS-08164525) were developed using Generalized Additive Models (GAMs)
relating nutrient concentration to river discharge, season, and time.
Separate models were fit at each station for each parameter and used to
predict nutrient concentrations for each day in the study period. GAMs
can be specified in a functionally similar manner to the commonly used
LOADEST \citep{cohn_validity_1992} or WRTDS \citep{hirsch_weighted_2010}
regression models and have been shown to produce reliable estimates of
nutrient and sediment loadings
\citep{wangLoadEstimationUncertainties2011, kroonRiverLoadsSuspended2012, kuhnert_quantifying_2012, robson_prediction_2015-1, hagemannEstimatingNutrientOrganic2016, mcdowell_implications_2021, biagi_novel_2022}.
GAMs are a semiparametric extension of generalized linear models where
the linear predictor is represented as the sum of multiple unknown
smooth functions and parametric linear predictors
\citep{wood_fast_2011}. Although the underlying parameter estimation
procedure of GAMs is substantially different than WRTDS, both the
functional form and results are demonstrated to be similar
\citep{beckNumericalQualitativeContrasts2017}. The use of GAMs over
other regression-based approaches was (1) the ability to easily explore
and incorporate different model terms, (2) the ability to incorporate
non-linear smooth function without explicit a priori knowledge of the
expect shape, and (3) the ability to specify a link function that
relates the expected value of the response to the linear predictors and
avoid data transformations.

GAMs were fit using the \emph{mgcv} package in R which makes available
multiple types of smooth functions with automatic smoothness selection
\citep{wood_fast_2011}. The general form of the model related
NO\textsubscript{3} or TP concentration to a long term tend, season,
streamflow, and two different antecedent discharge terms:

\begin{equation}\label{eq:1}
\begin{aligned}
g(\mu) &= \alpha + f_1(ddate) + f_2(yday) + f_3(log1p(Q)) + f_4(ma) + f_5(fa)  \\
y &\sim \mathcal{N}(\mu,\,\sigma^{2}),
\end{aligned}
\end{equation}

where \(\mu\) is the conditional expected NO\textsubscript{3} or TP
concentration, \emph{g()} is the log-link, \(\alpha\) is the intercept,
\emph{f\textsubscript{n}()} are smoothing functions. \emph{y} is the
response variable (NO\textsubscript{3} or TP concentration) modeled as
normally distributed with mean \(\mu\) and standard deviation
\(\sigma\). \emph{ddate} is the date converted to decimal notation,
\emph{yday} is numeric day of year (1-366), and \emph{log1p(Q)} is the
natural log of mean daily streamflow plus 1.

Moving average (\emph{ma}) is an exponentially smoothed moving average
that attempts to incorporate the influence of prior streamflow events on
concentration at the current time period
\citep{wangLoadEstimationUncertainties2011, kuhnert_quantifying_2012, zhang_improving_2017},
and expressed in \citet{kuhnert_quantifying_2012} as:

\begin{equation}\label{eq:2}
ma(\delta) = d{\kappa_{i-1}}+(1-\delta)\hat{q}_{i-1}\quad\text{and}\quad \kappa_{i}=\sum_{m=1}^{i}\hat{Q}_m,
\end{equation}

where \(\delta\) is the discount factor (here, set equal to 0.95),
\(\kappa_i\) is the cumulative flow (\emph{Q}) up to the \emph{i}th day.

Flow anomaly (\emph{fa}) is a unitless term that represents how wet or
dry the current time period is from a previous time period
\citep{vecchia_trends_2009, zhang_improving_2017}. Long-term flow
anomaly (\emph{ltfa}) is the streamflow over the previous year relative
to the entire period and calculated as described by
\citet{zhang_improving_2017}:

\begin{equation}\label{eq:3}
ltfa(t) = \bar{x}_{1\,year}(t) - \bar{x}_{entire\,period} 
\end{equation}

and the short-term flow anomaly (\emph{stfa}) calculated as the current
day flow compared to the preceding 1-month streamflow:

\begin{equation}\label{eq:4}
stfa(t) = x_{current\,day}(t) - \bar{x}_{1\,month}(t) 
\end{equation}

where \emph{x} are the averages of log-transformed streamflow over the
antecedent period (\emph{1-year}, \emph{1-month}, etc.) for time
\emph{t}. We used \emph{ltfa} in NO\textsubscript{3} models and
\emph{stfa} in TP models based on results from
\citet{zhang_improving_2017} demonstrating major improvements in
NO\textsubscript{x} regression models that incorporated \emph{ltfa} and
moderate improvements in TP regression models that incorporated
\emph{stfa}.

The calculation of model terms for the Lake Texana site were slightly
modified because daily loads are not a function of natural stream flow
processes alone, but of dam releases and nutrient concentrations at the
discharge point of the lake. \emph{Q}, \emph{ma}, and \emph{fa} terms
were calculated based on total gaged inflow from the 4 major tributaries
to the lake. Thin-plate regression splines were used for \emph{ddate},
\emph{log1p(Q)}, \emph{fa}, and \emph{ma}. A cyclic cubic regression
spline was used for \emph{yday} to ensure the ends of the spline match
(day 1 and day 366 are expected to match). First order penalties were
applied to the smooths of flow-based variables which penalize departures
from a flat function to help constrain extrapolations for high flow
measurements.

Left-censored data were not uncommon in this dataset. Several methods
are available to account for censored data. We transformed left-censored
nutrient concentrations to one-half the detection limit. Although this
simple approach can introduce bias
\citep{hornungEstimationAverageConcentration1990}, we considered it
acceptable because high concentrations and loadings are associated with
high-flow events and low-flow/low-concentration events will account for
a small proportion of total loadings \citep{mcdowell_implications_2021}.

Daily loads were estimated as the predicted concentration multiplied by
the daily streamflow. For the Navidad River (USGS-08164525) site, daily
loads at the dam were calculated from the discrete daily concentration
at the discharge point of the lake and corresponding reported daily
discharge from the dam. Flow-normalized loads were estimated similar to
WRTDS by setting flow-based covariates on each day of the year equal to
each of the historical values for that day of the year over the study
period \citep{hirsch_weighted_2010}. The flow-normalized estimate was
calculated as the mean of all the predictions for each day considering
all possible flow values. Standard deviations and credible intervals
were obtained by drawing samples from the multivariate normal posterior
distribution of the fitted GAM
\citep{woodConfidenceIntervalsGeneralized2006, marraCoveragePropertiesConfidence2012, mcdowell_implications_2021}.
Uncertainty in loads were reported as 90\% credible intervals developed
by drawing 1000 realizations of parameter estimates from the
multivariate normal posterior distribution of the model parameters. GAM
performance was evaluated using repeated 5-fold cross validation
\citep{burmanComparativeStudyOrdinary1989} and average Nash-Sutcliffe
Efficiency (NSE), r\textsuperscript{2} and percent bias (PBIAS) metrics
across folds were calculated for each model.

\hypertarget{linking-estuary-water-quality-to-hydrology-and-nutrient-loads}{%
\subsection{Linking Estuary Water Quality to Hydrology and Nutrient
Loads}\label{linking-estuary-water-quality-to-hydrology-and-nutrient-loads}}

To test if changes in freshwater inflow and nutrient loading had
explanatory effect on changes in estuary water quality a series of GAM
models were fit at each site relating parameter concentration to
temporal trends, inflow, and nutrient loads
\citep{murphyNutrientImprovementsChesapeake2022}:

\begin{equation}\label{eq:5}
\begin{aligned}
g(\mu) &= \alpha + f_1(ddate) + f_2(yday) \\
y &\sim \Gamma(\mu,\lambda),
\end{aligned}
\end{equation}

\begin{equation}\label{eq:6}
\begin{aligned}
g(\mu) &= \alpha + f_1(ddate) + f_2(yday) + f_3(Q) \\
y &\sim \Gamma(\mu,\lambda),
\end{aligned}
\end{equation}

\begin{equation}\label{eq:7}
\begin{aligned}
g(\mu) &= \alpha + f_1(ddate) + f_2(yday) + f_3(Q) + f_4(Load) \\
y &\sim \Gamma(\mu,\lambda),
\end{aligned}
\end{equation}

where \(\mu\) is the conditional expected response (nutrient
concentration), \emph{g()} is the log link, and response variable was
modeled as Gamma distributed with mean \(\mu\) and scale \(\lambda\).
\emph{f\textsubscript{1}(ddate)} is decimal date smoothed with a
thin-plate regression spline, \emph{f\textsubscript{2}(yday)} is the
numeric day of year smoothed with a cyclic cubic regression spline,
\emph{f\textsubscript{3}(Q)} is mean daily inflow (the combined
measurements from Lavaca River and Navidad River) and
\emph{f\textsubscript{4}(Load)} is the total NO\textsubscript{3} or TP
watershed load. The set of models specified for each water quality
response are in Table \ref{tab:estgammodels}.

\begin{table}[H]

\caption{\label{tab:estgammodels}Set of GAM models specified for each water quality parameter response.}
\centering
\begin{tabular}[t]{lll}
\toprule
\makecell[l]{Water Quality\\Response Parameter} & Model & Model Terms\\
\midrule
 & Temporal & s(ddate) + s(yday)\\

 & Flow & s(ddate) + s(yday) + s(Q)\\

\multirow{-3}{*}{\raggedright\arraybackslash TP} & Flow+Load & s(ddate) + s(yday) + s(Q) + s(TP Load)\\
\cmidrule{1-3}
 & Temporal & s(ddate) + s(yday)\\

 & Flow & s(ddate) + s(yday) + s(Q)\\

\multirow{-3}{*}{\raggedright\arraybackslash NO\textsubscript{\emph{x}}} & Flow+Load & s(ddate) + s(yday) + s(Q) + s(NO\textsubscript{3} Load)\\
\cmidrule{1-3}
 & Temporal & s(ddate) + s(yday)\\

 & Flow & s(ddate) + s(yday) + s(Q)\\

\multirow{-3}{*}{\raggedright\arraybackslash Chlorophyll-\emph{a}} & Flow+Load & s(ddate) + s(yday) + s(Q) + s(TP Load) + s(NO\textsubscript{3} Load)\\
\cmidrule{1-3}
 & Temporal & s(ddate) + s(yday)\\

 & Flow & s(ddate) + s(yday) + s(Q)\\

\multirow{-3}{*}{\raggedright\arraybackslash Dissolved Oxygen} & Flow+Load & s(ddate) + s(yday) + s(Q) + s(TP Load) + s(NO\textsubscript{3}  Load)\\
\cmidrule{1-3}
 & Temporal & s(ddate) + s(yday)\\

\multirow{-2}{*}{\raggedright\arraybackslash TKN} & Flow & s(ddate) + s(yday) + s(Q)\\
\bottomrule
\end{tabular}
\end{table}

Because streamflow and nutrient loads are tightly correlated, freshwater
inflow can mask signals from nutrient loads alone. Following the
methodology implemented by
\citet{murphyNutrientImprovementsChesapeake2022}, both streamflow and
nutrient loads were prepossessed to account for season and flow.
Freshwater inflow and nutrient loads were replaced by seasonally
adjusted log transformed inflow and flow-adjusted log transformed
nutrient loads obtained by fitting a GAM relating season (day of year)
to log transformed daily freshwater inflow values:

\begin{equation}\label{eq:8}
g(\mu) = \alpha + f_1(yday),
\end{equation}

and a GAM relating log transformed NO\textsubscript{3} or TP loads to
log transformed daily inflow:

\begin{equation}\label{eq:9}
g(\mu) = \alpha + f_1(log(Q)),
\end{equation}

where the response variables were modeled as normally distributed with
an identity link function. Response residuals from the respective GAM
models were used as \emph{Q} and \emph{Load} in Equation \ref{eq:6} and
Equation \ref{eq:7}.

This study used an information theoretic approach to evaluate if
nutrient loads and/or freshwater inflows provided evidence of effects on
water quality concentrations in Lavaca Bay. Model probabilities were
calculated and compared using the AIC\textsubscript{c} scores between
each group of temporal, flow, and flow+load models
\citep{burnhamAICModelSelection2011}. Improvements in model
probabilities provide evidence that the terms explain additional
variation in the response variable. If model probabilities were tied,
there wasn't evidence the more complicated model explains additional
variation in water quality.

\hypertarget{results}{%
\section{Results}\label{results}}

\hypertarget{watershed-nutrient-loads}{%
\subsection{Watershed Nutrient Loads}\label{watershed-nutrient-loads}}

Using evaluation criteria from
\citet{moriasiHydrologicWaterQuality2015}, predictive performance of
nutrient loading GAMs ranged from ``satisfactory'' to ``very good''
based on median NSE, r\textsuperscript{2}, and PBIAS metrics calculated
using 5-fold cross validation. Median NSE values were 0.34 (Lavaca
River) and 0.48 (Navidad River) for NO\textsubscript{3} loads and 0.81
(Lavaca River) and 0.91 (Navidad River) for TP loads. Median
r\textsuperscript{2} values were 0.70 (Lavaca River) and 0.87 (Navidad
River) for NO\textsubscript{3} loads and 0.93 (Lavaca River) and 0.99
(Navidad River) fopr TP loads. Median PBIAS values were 2.00 (Lavaca
River) and 10.90 (Navidad River) for NO\textsubscript{3} loads and -7.20
(Lavaca River) and -3.30 (Navidad River) for TP loads. Density plots of
metrics show similar distribution of values between sites for the same
parameter, with the exception r\textsuperscript{2} values for
NO\textsubscript{3} loads where USGS-08164000 showed a much larger
variance in values compared to USGS-08164525 (Figure \ref{fig:fig2}). TP
GAMS had higher average NSE and r\textsuperscript{2} values and lower
variance in metric values compared to NO\textsubscript{3}.

\begin{figure}\begin{adjustwidth}{-\extralength}{0cm}

{\centering \includegraphics[width=1\linewidth]{Schramm-Manuscript-2023_files/figure-latex/fig2-1} 

}

\end{adjustwidth}\caption[Density plots of goodness-of-fit metrics (NSE, r\textsuperscript{2}, and PBIAS) from repeated 5-fold cross validation between predicted nutrient loads from GAM models and measured nutrient loads]{Density plots of goodness-of-fit metrics (NSE, r\textsuperscript{2}, and PBIAS) from repeated 5-fold cross validation between predicted nutrient loads from GAM models and measured nutrient loads. Color indicates the tail probability calcualted from the empirical cumulative distribution of the goodness-of-fit metrics.}\label{fig:fig2}
\end{figure}

Predicted annual NO\textsubscript{3} and TP loads show considerable
variation, generally following patterns in discharge (Figure
\ref{fig:fig3}, Figure \ref{fig:fig4}). Flow-normalized TP loads at both
sites and flow-normalized NO\textsubscript{3} loads at USGS-08164000
indicate watershed based loads did not change much over time when
accounting for variation driven by streamflow (Figure \ref{fig:fig3}).
Flow-normalized loads at USGS-08164000 show small variation over time
with some decreases in NO\textsubscript{3} loads since 2013.

Aggregated across both sites, the mean annual NO\textsubscript{3} load
2005 through 2020 was 205,405 kg (126,867 kg - 341,569 kg, 90\% CI).
Annual NO\textsubscript{3} loads ranged from 12,574 kg in 2011 to
794,510 kg in 2007. Total annual TP loads ranged from 7,839 kg in 2011
to 595,075 kg in 2007. Mean annual TP loading from 2005 through 2020 was
182,673 kg (152,227 kg - 219,310 kg, 90\% CI). On average, USGS-08164525
accounted for 68\% of NO\textsubscript{3} loads and 59\% of TP loads
from 2005 through 2020. However, during periods of extreme drought the
Lavaca River (USGS-08164000) became the primary source of nutrient
loading in the watershed with the Navidad River only accounting for 15\%
and 25\% of NO\textsubscript{3} and TP loads in 2011 (Figure
\ref{fig:fig4}).

\begin{figure}

{\centering \includegraphics[width=1\linewidth]{Schramm-Manuscript-2023_files/figure-latex/fig3-1} 

}

\caption[Aggregated estimated annual and flow-normalized annual NO\textsubscript{3} and TP loads for USGS-08164000 and USGS-08164525]{Aggregated estimated annual and flow-normalized annual NO\textsubscript{3} and TP loads for USGS-08164000 and USGS-08164525.}\label{fig:fig3}
\end{figure}

\begin{figure}

{\centering \includegraphics[width=1\linewidth]{Schramm-Manuscript-2023_files/figure-latex/fig4-1} 

}

\caption[Comparison of delivered annual loads at USGS-08164000 and USGS-08164525]{Comparison of delivered annual loads at USGS-08164000 and USGS-08164525.}\label{fig:fig4}
\end{figure}

\hypertarget{linkages-between-water-quality-and-watershed-flows-and-loads}{%
\subsection{Linkages Between Water Quality and Watershed Flows and
Loads}\label{linkages-between-water-quality-and-watershed-flows-and-loads}}

\begin{figure}\begin{adjustwidth}{-\extralength}{0cm}

{\centering \includegraphics[width=1\linewidth]{Schramm-Manuscript-2023_files/figure-latex/fig5-1} 

}

\end{adjustwidth}\caption[Smoothed temporal trend component for water quality paramaters obtained from temporal estuary GAMs]{Smoothed temporal trend component for water quality paramaters obtained from temporal estuary GAMs.}\label{fig:fig5}
\end{figure}

GAMs did not identify significant changes in TP or DO concentrations at
any of the Lavaca Bay sites from 2005 through 2020 (Figure
\ref{fig:fig5}). The upper-bay site, TCEQ-13563, had a linear increase
in NO\textsubscript{x} concentration and and decrease in
chlorophyll-\emph{a} from 2005 through 2014. The mid-bay site,
TCEQ-13383, showed a periodic pattern in NO\textsubscript{x}
concentration that appeared similar to precipitation/inflow patterns, as
well as a post 2011 increase in TKN concentrations. No significant
long-term trends in concentrations were identified by GAMs for the
lower-bay TCEQ-13384 site.

Freshwater inflow provided additional explanation for changes in TP and
NO\textsubscript{\emph{x}} concentration at all of the Lavaca Bay sites
according to AIC\textsubscript{c} and model probability values (Table
\ref{tab:tab4}). TCEQ-13563, the site closest to the river outlet, was
the only site that had improvements in the explanations of DO and TKN
concentration with the inclusion of inflow. Both TCEQ-13563 and
TCEQ-13383, the mid-bay site, saw improvements in explanations for
variations in chlorophyll-\emph{a} with the inclusion of freshwater
inflow. The addition of nutrient loads (both TP and NO\textsubscript{3})
terms did not provide additional explanation for changes in
chlorophyll-\emph{a} or DO concentrations. Inclusion of TP loads
provided additional explanation of TP concentrations at the upper- and
mid-bay sites, TCEQ-13563 and TCEQ-13383. Inclusion of
NO\textsubscript{3} loads only provided marginal improvements in the
explanation of NO\textsubscript{\emph{X}} concentration at the lower-bay
TCEQ-13384 site.

\begin{table}[H]

\caption{\label{tab:tab4}Estuary GAM AIC\textsubscript{c} values and associated model probabilities (in parenthesis). Models with the highest probability for each site and water quality parameter combination are bolded and italicized for emphasis.}
\centering
\begin{tabular}[t]{ll>{}l>{}l>{}l}
\toprule
Parameter & Site & Temporal & Flow & Flow + Load\\
\midrule
 & TCEQ-13383 & -152.1 (0.03) & -156.1 (0.24) & \em{\textbf{-158.2 (0.72)}}\\

 & TCEQ-13384 & -194.4 (0.03) & \em{\textbf{-200.2 (0.49)}} & -200.2 (0.49)\\

\multirow{-3}{*}{\raggedright\arraybackslash TP} & TCEQ-13563 & -145.3 (0) & -156.6 (0.41) & \em{\textbf{-157.3 (0.59)}}\\
\cmidrule{1-5}
 & TCEQ-13383 & -218.9 (0) & \em{\textbf{-244.8 (0.5)}} & -244.8 (0.5)\\

 & TCEQ-13384 & -263.4 (0) & -311.7 (0.48) & \em{\textbf{-311.9 (0.52)}}\\

\multirow{-3}{*}{\raggedright\arraybackslash NO\textsubscript{\emph{x}}} & TCEQ-13563 & -175.1 (0) & \em{\textbf{-190.2 (0.5)}} & -190.2 (0.5)\\
\cmidrule{1-5}
 & TCEQ-13383 & 279.7 (0.18) & \em{\textbf{278.1 (0.41)}} & 278.1 (0.41)\\

 & TCEQ-13384 & \em{\textbf{268.2 (0.33)}} & 268.2 (0.33) & 268.2 (0.33)\\

\multirow{-3}{*}{\raggedright\arraybackslash Chlorophyll-\emph{a}} & TCEQ-13563 & 289.5 (0.08) & \em{\textbf{286.1 (0.46)}} & 286.1 (0.46)\\
\cmidrule{1-5}
 & TCEQ-13383 & \em{\textbf{42.2 (0.66)}} & 43.5 (0.34) & -\\

 & TCEQ-13384 & \em{\textbf{34.3 (0.57)}} & 34.8 (0.43) & -\\

\multirow{-3}{*}{\raggedright\arraybackslash TKN} & TCEQ-13563 & 31.1 (0.22) & \em{\textbf{28.7 (0.78)}} & -\\
\cmidrule{1-5}
 & TCEQ-13383 & \em{\textbf{146.4 (0.34)}} & 146.4 (0.34) & 146.5 (0.32)\\

 & TCEQ-13384 & \em{\textbf{135.9 (0.47)}} & 137 (0.27) & 137 (0.27)\\

\multirow{-3}{*}{\raggedright\arraybackslash DO} & TCEQ-13563 & 138.3 (0.25) & \em{\textbf{137.2 (0.43)}} & 137.8 (0.32)\\
\bottomrule
\end{tabular}
\end{table}

\begin{figure}\begin{adjustwidth}{-\extralength}{0cm}

{\centering \includegraphics[width=1\linewidth]{Schramm-Manuscript-2023_files/figure-latex/fig6-1} 

}

\end{adjustwidth}\caption[Estimated effects of mean daily inflow residuals on mean TP, NO\textsubscript{\emph{x}}, chlorophyll-\emph{a}, TKN, and DO concentrations in Lavaca Bay obtained from flow estuary GAMs]{Estimated effects of mean daily inflow residuals on mean TP, NO\textsubscript{\emph{x}}, chlorophyll-\emph{a}, TKN, and DO concentrations in Lavaca Bay obtained from flow estuary GAMs.}\label{fig:fig6}
\end{figure}

GAMs showed increases in freshwater inflow resulted in nearly linear
increases in TP and NO\textsubscript{\emph{x}} concentration at all
three sites (Figure \ref{fig:fig6}). At the upper-bay TCEQ-13563 site,
GAMs showed increases in freshwater inflow initially increased
chlorophyll-\emph{a} and DO concentration but concentrations leveled and
potentially decreased at higher flows. The mid-bay TCEQ-13383 site
showed a nearly linear increased in chlorophyll-\emph{a} concentration
in response to increases freshwater inflow. Freshwater flow did not have
significant effects on chlorophyll-\emph{a}, TKN, or DO at the lower-bay
TCEQ-13384 site.

Increased TP loads resulted in nearly linear increases of TP
concentration at the upper- and mid-bay sites, TCEQ-13563 and TCEQ-13383
respectively (Figure \ref{fig:fig7}). The relative effect size appeared
to much smaller than the effect of freshwater inflow alone. Increased
NO\textsubscript{3} loads only showed an effect at the lower-bay
TCEQ-13384 site. The effect was quite small compared to streamflow and
provided only small improvements to the model (Table \ref{tab:tab4}). As
noted above, nutrient loadings did not provide any explanation in
changes in the remaining assessed water quality parameters.

\begin{figure}

{\centering \includegraphics[width=1\linewidth]{Schramm-Manuscript-2023_files/figure-latex/fig7-1} 

}

\caption[Estimated effects of nutrient load residuals on TP and NO\textsubscript{\emph{x}} concentrations in Lavaca Bay obtained from flow+load estuary GAMs]{Estimated effects of nutrient load residuals on TP and NO\textsubscript{\emph{x}} concentrations in Lavaca Bay obtained from flow+load estuary GAMs.}\label{fig:fig7}
\end{figure}

\hypertarget{discussion}{%
\section{Discussion}\label{discussion}}

TP and NO\textsubscript{3} loadings from the Lavaca Bay watershed showed
high inter-annual variability tied with changes in discharge. There is
little evidence for changes in flow-normalized TP loads in either
rivers. There is some evidence of recent decreases in flow-normalized
NO\textsubscript{3} loads in the Lavaca River. Although there is no work
directly correlating water quality planning and implementation efforts
in the watershed to water quality outcomes, efforts to increase
agricultural producer participation in the watershed have been ongoing
since 2016
\citep{schramm_lavaca_2018, bertholdDirectMailingEducation2021}. The
decrease in flow-normalized NO\textsubscript{3} loads could be a
reflection of those collective efforts but further data collection and
research is required to support that statement.

Converted to average annual yield, the estimates of annual TP loads at
Lavaca River (USGS-08164000) are within the ranges in previous published
studies (Table \ref{tab:tpcomp};
\citep{dunnTrendsNutrientInflows1996, rebich_sources_2011, omaniEstimationSedimentNutrient2014, wise_spatially_2019}).
It isn't obvious why TP estimates in
\citet{dunnTrendsNutrientInflows1996} were notably lower. Given that
none of the studies identify substantially sized trends in TP, it is is
possible that the period used in \citet{dunnTrendsNutrientInflows1996}
was drier on average than the other studies. The SPARROW models used in
\citet{rebich_sources_2011} and \citet{wise_spatially_2019} utilize a
version of LOADEST in the underlying load estimation procedure, so a
difference due to methodology alone is unlikely.

\begin{table}[H]

\begin{threeparttable}
\caption{\label{tab:tpcomp}Mean estimates of annual TP yield in the Lavaca River watershed in published studies.}
\centering
\begin{tabular}[t]{lllll}
\toprule
Parameter & \makecell[c]{Reported Yield\\(kg$\cdot$km\textsuperscript{-2}$\cdot$year\textsuperscript{-1})} & Approach & Time Period & Reference\\
\midrule
TP & 35.2 (28.8, 43.3)\textsuperscript{*} & GAM & 2005-2020 & This work\\
TP & 45.2 & SPARROW & 2000-2014 & \citet{wise_spatially_2019}\\
TP & 42 & SWAT & 1977-2005 & \citet{omaniEstimationSedimentNutrient2014}\\
TP & 20.81-91.58\textsuperscript{\dag} & SPARROW & 1980-2002 & \citet{rebich_sources_2011}\\
TP & 28.9 & LOADEST & 1972-1993 & \citet{dunnTrendsNutrientInflows1996}\\
\bottomrule
\end{tabular}
\begin{tablenotes}
\small
\item [*] Values represent the mean of annual point estimates, lower and upper 95\% credible intervals.
\item [\dag] A single point estimate was not reported, these value represent the range depicted on the choropleth map provided in the report.
\end{tablenotes}
\end{threeparttable}
\end{table}

Cross-validation of the GAM loading models indicated that GAMs performed
well on average at predicting daily nutrient loading values. The
variance in scores was very high indicating subsets of values were
problematic at characterizing functional relationships between nutrients
and predictors. Because all of the water quality data for these two
locations in the TCEQ databases were ambient water quality data,
collected to be representative of typical flow conditions, there were
few data at the highest portions of the flow-duration curve. It was
beyond the scope of the current study to evaluate the subsets of
cross-validation data and scores. However, the cross-validation
procedure is indicative that more robust sampling would be beneficial
for reducing prediction variance. Supplementary flow-biased monitoring
targeting storm- or high-flow conditions is recommended here to improve
the precision of GAM predictions
\citep{horowitzEvaluationSedimentRating2003, snelderEstimationCatchmentNutrient2017}.

The non-linear temporal water quality trends identified using GAMs
differed slightly from the trends identified by
\citet{bugica_water_2020}. This is not unexpected due to the different
time periods, different methodology, and generally small slopes
identified for most of the significant water quality parameters in prior
work. The trend in DO and cholorophyll-\emph{a} concentrations are
stable in comparison to other Texas estuaries that are facing larger
demands for freshwater diversions, higher population growth, and more
intense agricultural production
\citep{wetzWaterQualityDynamics2016, bugica_water_2020}. The trend of
increasing NO\textsubscript{\emph{x}} concentration at the upper-bay
TCEQ-13563 site and recent increases in TKN concentration at the mid-bay
TCEQ-13383 site are concerning due to the nitrogen limitation identified
in many Texas estuaries
\citep{gardnerNitrogenFixationDissimilatory2006, houTransformationFateNitrate2012, doradoUnderstandingInteractionsFreshwater2015, paudelRelationshipSuspendedSolids2019, wetz_exceptionally_2017}
and the relatively low ambient concentrations observed in Lavaca Bay.

The strong positive effect of freshwater inflow on
NO\textsubscript{\emph{x}}, TKN, and TP are suggestive of nonpoint
watershed sources, consistent with watershed uses and with other studies
relating freshwater inflow with nutrient concentrations in Lavaca Bay
and other estuaries
\citep{russell_effect_2006, caffreyHighNutrientPulses2007, peierlsNonmonotonicResponsesPhytoplankton2012, palmerImpactsDroughtsLow2015, ciraPhytoplanktonDynamicsLowInflow2021}.
Inflow had a non-linear relationship with TKN at the two upstream sites,
with TKN increasing as freshwater inflow transitioned from low to
moderate levels. At higher freshwater inflows, the effect was
attenuated, possibly indicating a flushing effect at higher freshwater
inflow. No relationship between TKN and freshwater inflow were observed
at TCEQ-13384 located in the lower reach of Lavaca Bay. Tidal flushing
from Matagorda Bay could be responsible for diluting TKN and acting as a
control on the effects of freshwater inflow in lower reaches of Lavaca
Bay. \citet{russell_effect_2006} also suggested the processing of
organic loads in the upper portions of Lavaca Bay reduces the transport
of nutrients into the lower reaches of the Bay.

Freshwater inflow had a strong positive effect on chlorophyll-\emph{a}
at the upper- and mid-bay sites. The upper-bay site, TCEQ-13563, showed
decreases in chlorophyll-\emph{a} at the highest freshwater inflow
volumes. Freshwater flushing or increases in turbidity are associated
with decreases in chlorophyll-\emph{a} in other estuaries
\citep{peierlsNonmonotonicResponsesPhytoplankton2012, cloernPhytoplanktonPrimaryProduction2014}.
No relationships between inorganic nitrogen or TP loadings with
chlorophyll-\emph{a} were observed. Due to the lack of TKN loading
information, no assessment between organic nitrogen loads and
chlorophyll-\emph{a} were possible.

Although other studies have identified complex relationships between
estuary nutrient concentrations, nutrient loading and
chlorophyll-\emph{a} concentrations in Texas estuaries
\citep{ornolfsdottirNutrientPulsingRegulator2004, doradoUnderstandingInteractionsFreshwater2015, ciraPhytoplanktonDynamicsLowInflow2021, tominackVariabilityPhytoplanktonBiomass2022},
this study specifically used flow-adjusted freshwater derived nutrient
loads to parse out contributions from changes in nutrient loadings while
accounting for variations in load due to flow. Loading GAMs indicated no
evidence of changes in flow-normalized TP loads in either river, and no
changes in flow-normalized NO\textsubscript{3} loads in the Navidad
River. The small changes in flow-normalized NO\textsubscript{3} loads in
the Lavaca River are probably masked under most conditions by discharge
from the Navidad River. Given the relatively small variation in
flow-normalized loads, it can be expected that they would contribute
little to the variance in downstream water quality.

GAMs did not identify responses in DO concentration to inflows or
nutrient loads. The seasonality term in the temporal GAM models
explained a substantial amount of DO variation at all of the sites.
Responses of estuary metabolic processes and resulting DO concentrations
can be quite complicated and often locally specific
\citep{caffreyFactorsControllingNet2004}. While the lack of total
nitrogen or TKN loading data hinders interpretation, the large seasonal
effect on DO suggests physical factors play an important role and should
be included in future models. Prior work suggests that Lavaca Bay may
not be limited by nutrients alone, with high turbidity or nutrient
processing in upper portions of the Bay or intertidal river limiting
production \citep{russell_effect_2006}. Finally, it is reasonable to
assume that fluctuations in DO may not occur immediately in response to
nutrient pulses or freshwater inflow. Work has has shown that various
water quality parameters may have lagged effects lasting days or even
months following storms and large discharge events
\citep{mooney_watershed_2012, wetzExtremeFutureEstuaries2013, bukaveckas_influence_2020, walkerTimescalesMagnitudeWater2021}.
However, our work only evaluates responses to loading and inflows
occuring the day of water quality observations.

\hypertarget{conclusion}{%
\section{Conclusion}\label{conclusion}}

GAM models appear to provide reliable estimates of nutrient loads in the
Lavaca Bay watershed. However, additional flow-biased data collection
efforts would decrease the prediction variance and improve accuracy at
critical high flow events. Ongoing projects will fill data gaps for
total nitrogen and TKN loading. This study, consistent with others along
the Texas coast, found strong effects of freshwater flow on nutrient and
chlorophyll-\emph{a} concentrations. DO concentrations, dominated by
seasonal effects, did not show strong direct responses to freshwater
flow. Small variance in flow-adjusted nutrient loads indicates that (1)
there have been limited changes in non-point sources of nutrients and
(2) that there isn't strong evidence that those small changes have had
effects on chlorophyll-\emph{a} or dissolved oxygen in Lavaca Bay.
Although the study did not identify strong responses to changes in
nutrient loading, this does provide a baseline assessment for future
water quality management activities in the watershed.

%%%%%%%%%%%%%%%%%%%%%%%%%%%%%%%%%%%%%%%%%%

\vspace{6pt}

%%%%%%%%%%%%%%%%%%%%%%%%%%%%%%%%%%%%%%%%%%
%% optional

% Only for the journal Methods and Protocols:
% If you wish to submit a video article, please do so with any other supplementary material.
% \supplementary{The following supporting information can be downloaded at: \linksupplementary{s1}, Figure S1: title; Table S1: title; Video S1: title. A supporting video article is available at doi: link.}

%%%%%%%%%%%%%%%%%%%%%%%%%%%%%%%%%%%%%%%%%%

\funding{This project was funded in part by a Texas Coastal Management
Program grant approved by the Texas Land Commissioner, providing
financial assistance under the Coastal Zone Management Act of 1972, as
amended, awarded by the National Oceanic and Atmospheric Administration
(NOAA), Office for Coastal Management, pursuant to NOAA Award
No.~NA21NOS4190136. The views expressed herein are those of the
author(s) and do not necessarily reflect the views of NOAA, the U.S.
Department of Commerce, or any of their subagencies}



\dataavailability{Data and code are openly available in Zenodo at
\url{https://doi.org/10.5281/zenodo.7330754}.}

\acknowledgments{The author extends thanks to Dr.~Mike Wetz (Harte
Research Institute, Texas A\&M Corpus Christi), Chad Kinsfather and
Partick Brzozowski (Lavaca-Navidad River Authority), Brian Koch (Texas
State Soil and Water Conservation Board), Bill Balboa (Matagorda Bay
Foundation), Jason Pinchbeck (Texas General Land Office) and the Lavaca
Bay Foundation for supporting development of this project and providing
valuable feedback.}

\conflictsofinterest{The author declares no conflict of interest. The
project sponsors had no role in the design of the study; in the
collection, analyses, or interpretation of data; in the writing of the
manuscript, or in the decision to publish the results.}

%%%%%%%%%%%%%%%%%%%%%%%%%%%%%%%%%%%%%%%%%%
%% Optional

%% Only for journal Encyclopedia
%\entrylink{The Link to this entry published on the encyclopedia platform.}


%%%%%%%%%%%%%%%%%%%%%%%%%%%%%%%%%%%%%%%%%%
%% Optional
%%%%%%%%%%%%%%%%%%%%%%%%%%%%%%%%%%%%%%%%%%
\begin{adjustwidth}{-\extralength}{0cm}

%\printendnotes[custom] % Un-comment to print a list of endnotes


\reftitle{References}

% Please provide either the correct journal abbreviation (e.g. according to the “List of Title Word Abbreviations” http://www.issn.org/services/online-services/access-to-the-ltwa/) or the full name of the journal.
% Citations and References in Supplementary files are permitted provided that they also appear in the reference list here.

%=====================================
% References, variant A: external bibliography
%=====================================
\externalbibliography{yes}
\bibliography{mybibfile.bib}

% If authors have biography, please use the format below
%\section*{Short Biography of Authors}
%\bio
%{\raisebox{-0.35cm}{\includegraphics[width=3.5cm,height=5.3cm,clip,keepaspectratio]{Definitions/author1.pdf}}}
%{\textbf{Firstname Lastname} Biography of first author}
%
%\bio
%{\raisebox{-0.35cm}{\includegraphics[width=3.5cm,height=5.3cm,clip,keepaspectratio]{Definitions/author2.jpg}}}
%{\textbf{Firstname Lastname} Biography of second author}

% For the MDPI journals use author-date citation, please follow the formatting guidelines on http://www.mdpi.com/authors/references
% To cite two works by the same author: \citeauthor{ref-journal-1a} (\citeyear{ref-journal-1a}, \citeyear{ref-journal-1b}). This produces: Whittaker (1967, 1975)
% To cite two works by the same author with specific pages: \citeauthor{ref-journal-3a} (\citeyear{ref-journal-3a}, p. 328; \citeyear{ref-journal-3b}, p.475). This produces: Wong (1999, p. 328; 2000, p. 475)

%%%%%%%%%%%%%%%%%%%%%%%%%%%%%%%%%%%%%%%%%%
%% for journal Sci
%\reviewreports{\\
%Reviewer 1 comments and authors’ response\\
%Reviewer 2 comments and authors’ response\\
%Reviewer 3 comments and authors’ response
%}
%%%%%%%%%%%%%%%%%%%%%%%%%%%%%%%%%%%%%%%%%%
\PublishersNote{}
\end{adjustwidth}


\end{document}
